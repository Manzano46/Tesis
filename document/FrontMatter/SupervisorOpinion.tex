\begin{opinion}	
	
En los últimos años en nuestra facultad se han desarrollado varias tesis de licenciatura que tributan a agilizar el proceso de solución de un VRP, y aunque cada una de ellas han resuelto una arista del problema no era posible usarlas para resolver problemas, porque cada una existía independientemente de las demás.  Gracias al trabajo de Rodrigo, es posible usar los resultados de todas esas tesis en un único producto.

Uno de los méritos del trabajo es que Rodrigo tuvo que leer, entender y modificar más de 10 000 líneas de código en un lenguaje completamente distinto a todo lo conocido por él hasta ese momento.  Eso lo hizo de una manera excelente, al punto de poder discutir con los autores (de tú a tú) sobre cualquier aspecto del código.

A mi juicio, esto demuestra que Rodrigo es capaz de utilizar los conocimientos adquiridos durante la carrera de una manera creativa, puede identificar y solucionar problemas que aparezcan durante la investigación y ciertamente puede adquirir de manera autodidacta los conocimientos y habilidades que necesite.

Por todo lo anterior creo que estamos en presencia de un excelente científico de la computación, que al igual que el trabajo, merece la máxima calificación.


\vspace{1cm}


\begin{flushright}
	\underline{\hspace{6.5cm}}\\
	MSc. Fernando Raúl Rodríguez Flores
	
	Facultad de Matemática y Computación
	
	Universidad de la Habana
	
	Noviembre, 2022
\end{flushright}

\end{opinion}